%%% Testing:spacing
\documentclass[a4paper,12pt]{article}
\usepackage[utf8]{inputenc}
\usepackage[ngerman]{babel}

\hoffset=0pt
\voffset=0pt
\oddsidemargin=0pt
\topmargin=0pt
\headheight=0pt
\headsep=0pt

\usepackage[left=2cm, right=2cm, top=1.5cm, bottom=2cm, ]{geometry} 
\usepackage{tikz}
\usepackage{graphicx}
\usepackage{esvect}
\usepackage{sistyle}
\usepackage{textpos}

%for code
\usepackage{listings}
\usepackage{color}

\definecolor{dkgreen}{rgb}{0,0.6,0}
\definecolor{gray}{rgb}{0.5,0.5,0.5}
\definecolor{mauve}{rgb}{0.58,0,0.82}

\lstset{frame=tb,
  language=Java,
  aboveskip=3mm,
  belowskip=3mm,
  showstringspaces=false,
  columns=flexible,
  basicstyle={\small\ttfamily},
  numbers=none,
  numberstyle=\tiny\color{gray},
  keywordstyle=\color{blue},
  commentstyle=\color{dkgreen},
  stringstyle=\color{mauve},
  breaklines=true,
  breakatwhitespace=true,
  tabsize=3,
  texcl=true
}

\setlength{\TPHorizModule}{50pt}
\setlength{\TPVertModule}{\TPHorizModule}

\pagestyle{empty}

\title{Dummi}

\begin{document}


\setlength{\parindent}{0pt}
\setlength{\parskip}{30pt}
\setlength{\baselineskip}{20pt}


\begin{lstlisting}
// Syntax ""= in muss geschrieben werden []=wahlweise
int x, y; //deklaliert x und y als intiger
int x = 5, y = 7; //deklaliert und speicher die werte dazu
final int x = 8 //bleibt immer 8
x = X + 1
// 3.14 double
//3.14f float
// 13E4  13*e4

float f; int i;
f=i; //erlaubt
i=f; //verboten
i=(int)f; //erlaubt, schneidet nachkommazeilen oder mehr ab
f=1.0; //verboten weil 1.0 double ist
f=1.0f; //ok
//double + int //double
//float + int //float
//short + short //int
//if (n!=0) x = x/n; // if (True) mach x =...

if (x>y)
	max = x;
else 
	max = y;
// syntax "if" "(" Expresion ")" Statment ["else" Statment]
if (x>y){    
// fuer mehr Statements {} nennt sich Block
	max = x;
	System.out.println(x);
}
else 
	max = y; 
int x=0; // muss int oder float sein
x++ //x=1 x++=>x+= oder x=x+1
x-- // x=x-1 
== //gleich
!= // ungleich
> //grösser
< //kleiner
>= //grööser oder gleich
<= // kleiner oder gleich
&& // und-Verknüpfung if (... and ... ==True) mach ...
|| // oder-Verknüpfung
!x //nicht-Verknüpfung
x==True 
!x==False
boolean p, q;
	p=false;
	q = x<0;
whileschleife
i = 1;
sum = 0;
while (i<=n){   //schleifenbedingung //solange i<= n tu das...
	sum = sum + i; //Schleifenrumpf
	i += 1;   // "  "
}
do-schleife
int n = In.readInt(); //Zahl von Tastatur
do {
	System.out.println(n%10);  // mach das... und wiederhole wenn..
	n=n/10;
} while (n>0);
for-Schleifen
sum = 0;
for (i = 1; i <= n; i++ ) // 1. bestimmung einer Variablen, 
	sum = sum + 1;		// 2.voraussetzung zur weiterführung 
				// 3. veränderung der Vriable
		    // "for" "(" [ForInit] ";" [Expression] ";" [ForUpdate]";")
			   // ForInit = Assigment ("," Assigment)
	int = 0;	   //         = Zahlen Typ deklarieren | Type n =.. ("," n=..)
               // ForUpdate = Assigment ("," Assigment)
			 
for (;;){} //endlosschelife
	if (..)break;  // Beendet die Schleife in der es sich befiendet

hours_K = JOptionPane.showInputDialog("frage"); 
//öffnet interaktive Fenster und speichert antwort in variable (parse geht auch)
args[0].equals("String"); //vergleicht Input/Varible mit String

void //nimmt keinen wert an. die methode nimmt keinen wert an kann auch nicht zurück geben

static void "methodname"(){    //methodealggm.
}

static void numbers(int x,int y){     //methodenbeispiel
	if (x>y) system.out.println(x);
	else system.out.ptintln(y);
}

static int max(int y, int x){ //Funktion, kein void, gib eine Wert zurück, muss mit return enden
	if (x>y) return y;
	else return x;
}
 return //kann als break verwendet werden, springt aus der methode

int[] a;  //Array a as int
float[] b; //Array b as float

a = new int[5];  //how many varibles can fit in
b = new float[10]; 

int[][] a; //zweidimensionales array
a = new int[3][4];
int[][] a = {{1,3,5},{3,5,9},etc}

int[][] a = new int[3][]; //für unterschiedliche längen. (sehr selten brauchbar)
a[0] = new int[4];
a[1] = new int[2];
etc.

b = null; //can't find the array of b anymore
a = b; //greifen jetzt auf den gleichen Array zu, kann über beide variabeln verändert werden

b = (int[]) a.clone(); //koopiert a in b reine
a.lenght //die länge von a +1


char ch = "\u0041" //"\u0041" entspricht "A" im ascicode
		   // 0*16e3 + 0*16e2 + 4*16e1 + 1*16e0 A-F entsprechen 11-16 im 16 zahlensystem

char bonni = 'c';
int i = bonni;  //chars können integers zugewiesen werden

10 + (bonni - '0') //ergebniss int, bonni=99 '0'=48 => (99-48) + 10

if (Character.isLetter(bonni)) //true wenn bonni ein Unicode-Buchstabe ist
"" //for Strings
'' //for chars

if (s.equals("Hellow")) //kann nurnso direkt verglichen werden

//===================================================
class Date{         //definition einer Variable
	int day;
	String month;
	int year;
}

\end{lstlisting}

\end{document}
