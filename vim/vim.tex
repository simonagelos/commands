%%% Testing:spacing
\documentclass[12pt]{article}


%\usepackage[showboxes]{textpos}

\hoffset=0pt
\voffset=0pt
\oddsidemargin=0pt
\topmargin=0pt
\headheight=0pt
\headsep=0pt

\usepackage[left=2cm, right=2cm, top=1.5cm, bottom=2cm, ]{geometry} 
\usepackage{tikz}
\usepackage{graphicx}
\usepackage{esvect}
\usepackage{sistyle}
\usepackage{textpos}

\setlength{\TPHorizModule}{50pt}
\setlength{\TPVertModule}{\TPHorizModule}

\pagestyle{empty}
\usepackage[linktocpage=true]{hyperref}

\hypersetup{%
unicode,pdffitwindow,
pdfkeywords = {pdf, LaTeX, hyperref, thumbnails}, 
pdfauthor = {Brüder Grimm},
bookmarksopen = true,
bookmarksnumbered = true,
pdfcenterwindow=true,
pdffitwindow = true,
pdfstartview=FitBV,
pdfcreator = {pdflatex},
colorlinks=true, breaklinks=true, %
urlcolor=magenta, % color for \url
filecolor=cyan, % color for file
linkcolor=black, % color for \ref
citecolor=magenta, %color for \cite
menucolor=darkblue,
breaklinks=true,anchorcolor=green
}

\title{Dummi}

\begin{document}


\setlength{\parindent}{0pt}
\setlength{\parskip}{30pt}
\setlength{\baselineskip}{20pt}

% Make sure that all text appears precisely where it ought to.  The
% above dimensions mean that text appears in round-number locations.
%
% The \showbox puts detailed calculations in the log file
%
% Output of 'dvireport -F -up t5.dvi':
% c 97[a] 0,20pt
% c 121[y] 4.72223,20pt
% c 98[b] 0,28.8889pt
% c 121[y] 5.27777,28.8889pt
% c 99[c] 50,76.25pt
% c 100[d] 0,70pt
% c 121[y] 5.55556,70pt
%
% The most important thing is that `a' and `d' are 50pt
% apart, \parskip+\baselineskip
%
% This test may not currently be working (2005 August 30).  There's an
% extra 30pt (\parskip) appearing before the by and cy in their boxes,
% which clearly isn't present in the t5.correct.dvi.  Ought it to be
% there?  Have I done something (when?) which has caused this to
% spuriously appear?


\showboxdepth=8
\showboxbreadth=8
%\documentclass[12pt]{article}

%\usepackage[left=2cm, right=2cm, top=1.5cm, bottom=2cm, ]{geometry} 
%\usepackage{tikz}
%\usepackage{graphicx}


\begin{itemize}

    \item h moves the cursor one character to the left.w
    \item k moves the cursor up one line.
    \item j moves the cursor down one line.
    \item l moves the cursor one character to the right.
    \item 0 moves the cursor to the beginning of the line.
    \item \^ move to first non-block character in line. 
    \item \$ moves the cursor to the end of the line.
    \item w move forward one word.
    \item b move backward one word.
    \item G move to the end of the file.
    \item äs move to next misspelled word
    \item \$s move to last misspelled word
    \item gg move to the beginning of the file.
    \item line + G or gg moves curse to line.
    \item `. move to the last edit.
    \item A insert at the end of the line
    \item e  Puts the cursor at the end of a word
    \item ctrl + u  fast upwards
    \item ctrl + d  fast downwards
    \item ctrl + e  scroll down without cursor
    \item ctrl + p in commandmode will go history upwards
    \item crtl + n "                            " down
    \item n move the cursor to the next instance of the text from the last search. this will wrap to the beginning of the document.
    \item n move the cursor to the previous instance of the text from the last search.
    \item d starts the delete operation.
    \item x delets current char
    \item dd will delete the hole line.
    \item dw will delete a word.
    \item d0 will delete to the beginning of a line.
    \item d\$ will delete to the end of a line.
    \item dgg will delete to the beginning of the file.
    \item dG will delete to the end of the file.
    \item u will undo the last operation.
    \item Ctrl-r will redo the last undo. \\
    \item gt - move to the next tab
    \item gT - move to the previous tab
    \item \#gt - move to a specific tab number (e.g. 2gt takes you to the second tab)
    \item =G - indents code below
    \item CTRL-W h - moves in window above, j k l h for other windows
    \item CTRL-] - in helpfile on maktfiles will open them
\end{itemize}
\begin{itemize}
    \item - /*text* search for *text* in the document, going forward.
    \item ?*text* search for *text* in the document, going backwards.
    \item :\%s/*text*/*replacement text*/g search through the entire document for *text* and puts  *replacement text infront*. without text/g it replace
    \item :\%s/text/replacment/g replace all text
    \item :\%s/*text*/*replacement text*/gc search through the entire document and *confirm* before replacing text.
    \item Crtl + V after selection Shift + I insert what you want for these line
    \item /+text to be searched | with n move forward trough words | N backwards
    \item shift + i + text replaces what was marked in visual mode with text 
    \item :nohl to stop marking text
    \item z= on misspelt word for suggestens
    \item zg add misspelt word to dictionary
    \item zw say word i wrong  
    \item s/textreplace/textsubtution [g\&](https://www.notion.so/g-4a820f8ef73f420580ed02d2b726782b) to replace text for one line and then all
    \item noremap (inoremap, nnoremap, vnoremap) + "keytopress" + "what i does"  ignores ever other map 
    \item autocmd "looks always what you write
    \item Filetype + "filetype"  in what filetype the folowing wil happen
    \item :! + command "makes commmand in terminal bvut beeing in vim
    \item gt switches tabs
    \item :setlocal spell spelllang=de sets spell checking in german local buffer with just :set it is global
    \item :set nospell turns off spell checking
    \item :"lineNumberStart","lineNumberEnd"s/\^/"replaceItem" /  puts on beggining of the lines between lineNUmberStart and Linenumberend the replaceItem, works with crl-V to
    \item q: shows commandhistory that is executable
    \item \^\$ represents brakes
\end{itemize}
\begin{itemize}
    \item K show documentation for thing on cursor, the commant in lua doesnt work yet 
    \item v highlight one character at a time.
    \item V highlight one line at a time.
    \item V< to move selected lines to left.
    \item V> to move selected lines to right.
    \item Ctrl-V highlight by columns.
    \item p paste text on the current line.
    \item P paste before the cursor.
    \item y yank text into the copy buffer.
    \item y\$ yank till the end of the line.
    \item `y\$` - Yank (copy) everything from the cursor to the end of the line.
    \item `y\^` - Yank (copy) everything from the cursor to the start of the line.\\
    \item "+y copys selected data intop clipboard \\
\end{itemize}
\begin{itemize}
    \item :w  safe the file
    \item :e filename - opens filename and closes opend file
    \item :tabedit file - opens a new tab and will take you to edit "file"
    \item :tabe "filename" opens file in same terminal 
    \item :tabm i move tab to i beggining with 0
    \item :tabm +i/-i move tab i related to position of tab
    \item :tabs - list all open tabs
    \item :tabclose - close a single tab
    \item :w filename safe with new name
    \item :q finish vim if no safe u have to ! for safe
    \item :q! quit without safe
    \item :split "file" splits window in half on top will be file
    \item :vslpit "file" "                      " left will be file
    \item \href{https://github.com/junegunn/vim-plug}{pluginmanager}
\end{itemize}
\begin{itemize}
    \item ZZ safe and quit
    \item crl + x + f filenamecomplishen in vim
    \item ;it textit{}
    \item ;ib textbf{}
    \item ;ig include graphics
    \item ;sec ;sec* ;sub ;sub*
    \item ;im itemize
    \item ;cen center
    \item ;tb table with [H]
    \item ;bg begin something
\end{itemize}

tabby movment 
\begin{itemize}
    \item "leader"tn next tab
    \item "leader"tp previous tab
    \item "leader"to closes all other tabs
    \item C-t opens new tab
    \item C-w closes tab 
    \item "leader"tmp moves tab to previous prosition
    \item "leader"tmn moves tab to next prosition
\end{itemize}

Custom/Plugs
\begin{itemize}

    \item  cdacac

\end{itemize}

lsp, codehelp
\begin{itemize}

    \item K shows manuel entry of hovered item
    \item <space>rn renames variable everywhere
    \item gr shows rwhere variable/function is used/defined
    \item <Leader$>$e or d shows error on line
    \item :Telescope notify shows history of notifications 
    \item "leader"v finding file with telescope
    \item "leader"s source file
\end{itemize}


\end{document}
