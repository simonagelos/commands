%%% Testing:spacing
\documentclass[12pt]{article}


\usepackage{multicol}
\usepackage{url}


% for code snippits
\usepackage{listings}

\lstset{
    numbersep=3pt,
    keywordstyle=\color{blue},
    commentstyle=\color{dkgreen},
    stringstyle=\color{mauve},
    breaklines=true,
    numbers=left,
    numberstyle=\scriptsize\color{black},
    frame=none,
    basicstyle = \small\ttfamily,
    breaklines=true
    breakatwhitespace=false,
    columns=flexible,
    xleftmargin=0.5cm,framesep=8pt,framerule=0pt,
    aboveskip=3mm,
    belowskip=3mm,
}


\hoffset=0pt
\voffset=0pt
\oddsidemargin=0pt
\topmargin=0pt
\headheight=0pt
\headsep=0pt

\usepackage[left=0cm, right=0cm, top=0cm, bottom=0cm, ]{geometry} 
\usepackage{tikz}
\usepackage{graphicx}
\usepackage{esvect}
\usepackage{sistyle}
\usepackage{textpos}

\usepackage{float}


\setlength{\TPHorizModule}{50pt}
\setlength{\TPVertModule}{\TPHorizModule}

\pagestyle{empty}

\title{Dummi}

\begin{document}


\setlength{\parindent}{0pt}
\setlength{\parskip}{30pt}
\setlength{\baselineskip}{20pt}




\centering
\begin{tabular}{| p{6.5cm} | p{6.5cm} | p{6.5cm} |}
    \hline
    Command & Fuction & Informations \\
    \hline
    sudo dpkg -i *.deb & install new kernil & do in directory with *.deb kernelfiles \\
    \hline
    ps & shows running processes & \\
    \hline
    ps -aux & show all running processes & \\
    \hline
    source .zshrc & loads file and settings from .bash\_alias are load & same vor zshalias idk name  \\
    \hline
    source .path & that inported library works & in directory with the Programm \\
    \hline 
    cp **. & cp path-setup to execute then source &\\
    \hline
    du -h filename & gibt grösse von filname aus & \\
    \hline
    sudo find "directory" -name '*.apk' & looks for file with .apk ending &\\
    \hline 
    sudo umount "full device name" & to unmount device & \\
    \hline 
    mkfs.fat -F32 /dev/PARTITION & to format partition &\\
    \hline 
    chmod +x "filename" & makes file executable &\\
    \hline
    upower -i `upower -e | grep 'BAT'` & to see batteryinformations &\\
    \hline
    mutool draw -R 90 -o 'new name'.pdf 'file'.pdf & to rotate clockwise 90 degre &\\
    \hline 
    wc -l "file" & counts lines in file &\\
    \hline
    ls -s & shows size of files &\\
    \hline
    pandoc input.md -o output.pdf & compiles Markdown files to pdf\\
    \hline
    apt-cache pkgnames "beginning of package name"& to see all awailable packages, in "" is optionel and shows just the packages starting with \\
    \hline
    pushd directory & go in directory \\
    \hline 
    popd & when pushd made first us can go back where u were befor \\
    \hline
    awk & usefull for for outputs \\
    \hline 
    xargs & takes output and puts behind the programm folwing \\
    \hline
    bluetoothctl & to connect to bluetooth devices \\
    \hline
    file "image" & shows some metadata like size and format \\
    \hline
    find . -type f -name "* *" | while read file; do mv "\$file" \${file// /\_}; done & finds files with space and replaces it with \\
    \hline
    perm directory & shows permition from directory \\
    \hline
     sudo update-alternatives --config python3 & to update python3 verion\\
    \hline 


\end{tabular}

* \~/schulmaterial/efi/EigeneProjekte/inputoutputflanagan \\
** \~/schulmaterial/efi/Eigene\_Projekte/inputOutputFlanagan/path-setup 


\centering
\begin{tabular}{| p{6.5cm} | p{6.5cm} | p{6.5cm} |}
    \hline
    \multicolumn{3}{|c|}{Bash "Syntax"} \\
    \hline
    code & meaning & information\\
    \hline
    \$"something" & with \$ u access something & \$1 gets first input\\
    \hline
    ("Function") & inside () something like find or pwd gos &\\
    \hline
    {"Command"} & Command stands for like 1\#\#*/ & in compileInfo it cuts the filename \\
    \hline
    \${parameter\#pattern} & shortest match from pattern in parameter from beginning is deleted &\\
    \hline 
    \${parameter\#\#pattern} & longest match from pattern in parameter from beginning is deleted & \${1\#\#*/} deletes everything from input bevor the last / \\
    \hline
    \${parameter\%pattern} & like \# just from end &\\
    \hline 
    \${parameter\%\%pattern} & like \#\# just from the end &\\
    \hline 
    " text " & everything in "" will be like String and no commands except \$,',backslash & echo "pdflatex textdummit.tex" output pdflatex textdummi.tex \\
    \hline
    ' command/text ' & similar to "" but can use variables & echo "my variable value is '\$variable' " \\
    \hline 
    variable=5 & declaring variable as integer 5 & works like in python \\
    \hline 
    == != + - * / \% & operator like in java but has also other use &\\
    \hline 
    mutool draw -R 90 -o out.pdf Tenses.pdf & to rotate clockwise by 90 degree &\\
    \hline 
    ln -fs 'any dircetory' 'direcatory name in pwd' & creates directory linkt the new directory like a portal & \\
    \hline 
    > & to redirect output for example in a file & overrides existing file \\
    \hline
    >> & appends output into new director/file & doesn't override existing file \\
    \hline


\end{tabular}

\lstset{language=sh}
\begin{lstlisting} 
if [condition]
then
    statement 
fi

if [condition 1]
then 
    Statement 1
elif [condition 2]
then
    Statement 2
else 
    Statement 3
fi

\end{lstlisting}


for more information of syntax
{\url{https://www.gnu.org/software/bash/manual/bash.html#index-insert_002dcomment-_0028M_002d_0023_0029}}

{\url{https://www.tutorialspoint.com/unix/unix-basic-operators.htm}}





\end{document}
