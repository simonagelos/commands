%%% Testing:spacing
\documentclass[12pt]{article}


\usepackage{multicol}
\usepackage{url}


% for code snippits
\usepackage{listings}

\lstset{
    numbersep=3pt,
    keywordstyle=\color{blue},
    commentstyle=\color{dkgreen},
    stringstyle=\color{mauve},
    breaklines=true,
    numbers=left,
    numberstyle=\scriptsize\color{black},
    frame=none,
    basicstyle = \small\ttfamily,
    breaklines=true
    breakatwhitespace=false,
    columns=flexible,
    xleftmargin=0.5cm,framesep=8pt,framerule=0pt,
    aboveskip=3mm,
    belowskip=3mm,
}


\hoffset=0pt
\voffset=0pt
\oddsidemargin=0pt
\topmargin=0pt
\headheight=0pt
\headsep=0pt

\usepackage[left=0cm, right=0cm, top=0cm, bottom=0cm, ]{geometry} 
\usepackage{tikz}
\usepackage{graphicx}
\usepackage{esvect}
\usepackage{sistyle}
\usepackage{textpos}

\usepackage{float}


\setlength{\TPHorizModule}{50pt}
\setlength{\TPVertModule}{\TPHorizModule}

\pagestyle{empty}

\title{Dummi}

\begin{document}


\setlength{\parindent}{0pt}
\setlength{\parskip}{30pt}
\setlength{\baselineskip}{20pt}




\centering
\begin{tabular}{| p{6.5cm} | p{6.5cm} | p{6.5cm} |}
    \hline
    Command & Fuction & Informations \\
    \hline
    sudo dpkg -i *.deb & install .deb file or programm \\
    \hline
    ps & shows running processes & \\
    \hline
    ps -aux & show all running processes & \\
    \hline
    source .zshrc & loads file and settings from .bash\_alias are load & same vor zshalias idk name  \\
    \hline
    source .path & that inported library works & in directory with the Programm \\
    \hline 
    cp **. & cp path-setup to execute then source &\\
    \hline
    du -h filename & gibt grösse von filname aus & \\
    \hline
    sudo find "directory" -name '*.apk' & looks for file with .apk ending &\\
    \hline 
    sudo umount "full device name" & to unmount device & \\
    \hline 
    mkfs.fat -F32 /dev/PARTITION & to format partition &\\
    \hline 
    chmod +x "filename" & makes file executable &\\
    \hline
    upower -i `upower -e | grep 'BAT'` & to see batteryinformations &\\
    \hline
    mutool draw -R 90 -o 'new name'.pdf 'file'.pdf & to rotate clockwise 90 degre &\\
    \hline 
    wc -l "file" & counts lines in file &\\
    \hline
    ls -s & shows size of files &\\
    \hline
    pandoc input.md -o output.pdf & compiles Markdown files to pdf & \\
    \hline
    apt-cache pkgnames "beginning of package name"& to see all awailable packages, in "" is optionel and shows just the packages starting with &\\
    \hline
    pushd directory & go in directory & \\
    \hline 
    popd & when pushd made first us can go back where u were befor &\\
    \hline
    awk & usefull for for outputs &\\
    \hline 
    xargs & takes output and puts behind the programm folwing &\\
    \hline
    bluetoothctl & to connect to bluetooth devices &\\
    \hline
    file "image" & shows some metadata like size and format &\\
    \hline
    find . -type f -name "* *" | while read file; do mv "\$file" \${file// /\_}; done & finds files with space and replaces it with &\\
    \hline
    perm directory & shows permition from directory &\\
    \hline
    sudo update-alternatives --config python3 & to update python3 verion & \\
    \hline 
    zip -r newfile.zip folder/ & zips folder in file & \\
    \hline
    jobs & shows process in backgrouns & \\
    \hline 
    kill \%"number" & kills process with this number & \\
    \hline
    sudo update-alternatives --displa java & shows optional java versions and current version & \\
    \hline 
    java -version & shows current java version & \\
    \hline
    javac -version & to see version of java compiler & \\
    \hline 
    sudo update-alternatives --config java & to change the java version & \\
    \hline
    sudo update-alternatives --config javac & to change the version of the java compiler & \\
    \hline
    sudo apt install openjdk-17-jre & -jdk for java compiler, installs version 17 of java/javac & \\
    \hline 
    file & shows info about a file & \\
    \hline
    lsusb & list usb devices & \\
\end{tabular}

\newpage

\begin{tabular}{| p{6.5cm} | p{6.5cm} | p{6.5cm} |}
    mutool draw -R 90 -o out.pdf Tenses.pdf & to rotate clockwise by 90 degree &\\
    \hline 
    ln -fs 'any dircetory' 'direcatory name in pwd' & creates directory linkt the new directory like a portal & \\
    \hline 
    > & to redirect output for example in a file & overrides existing file \\
    \hline
    >> & appends output into new director/file & doesn't override existing file \\
    \hline
    tmux & starts a tmux session &\\
    \hline
    tmux ls & lists tmux sessions & \\
    \hline 
    crtl + b , D & detaches tmux session &\\
    \hline
    crtl + b , p/n & go to previous/next window & \\
    \hline
    crtl + b, 1 0..0 & switch/select window by number & \\
    \hline
    tmux list-sessions & shows active lessions & \\
    \hline
    crtl + b, s & "" & \\
    \hline
    zip -r zipname.zip directory1 directory2 file1 ... & to zip files and directory with -r & \\
    \hline
    f2 or esc to enter BIOS/UEFI -> boot menu & to boot from a external device, than it will also schow up in grub &\\
    \hline 
    usb-device & list device conectet &\\
    \hline
    ls /etc/apt/sources.list.d & list of repositorys installed via add-apt-repository & \\
    \hline
    sudo rm /etc/apt/sources.list.d/name- from-above & to delet a repository without the exact ppa name &\\
    \hline
    apt-key list & list keys for ppas &\\
    \hline
    sudo do-release-upgrade & updates software to newest Ubuntuversion &\\
    \hline 
    sudo do-release-upgrade -c & to see what versions are available &\\
    \hline
    apt show "programm" & shows info about the programm &\\
    \hline
    gpg -d "file" & to dencrypt file &\\
    \hline 
    find . -type f -name '*pattern*' & will find all files in . folder with name pattern, u can change pattern to any patter &\\
    \hline
    find . -type f -name '*pattern*' -exec mv \{\} . \; & moves every founde file to . or whereever i want it &\\
    \hline 
    pdfposter -mA2 example.pdf example-A2-pdfposter.pdf & makes pdf in A2 foramt &\\
    \hline
    pdfinfo & schows info of pdf & \\
    \hline
    mount "device name" "mount point" & mounts device & mount points for example: /mnt/usedrive /media/usb /media/USB \\
    \hline
    fdisk -l & list all the devices and also usb discs & \\
    \hline 
    bluetoothctl scan on & scans for devices & can also just first bluetoothctl an then do alle the commands \\
    \hline 
    bluetoothctl discoverable on & make own devicse discoverable & \\
    \hline 
    bluetoothctl pair FC:69:47:7C:9D:A3 & pair with device, thats an example name, if already paired once this is not necesairy & \\
    \hline 
    bluetoothctl connect FC:69:47:7C:9D:A3 & to connect to device & \\
    \hline
    bluetoothctl paired-devices & shows paired devices &  more commands on \url{https://www.makeuseof.com/manage-bluetooth-linux-with-bluetoothctl/}\\
    \hline 
    wc & programm that can count lines, character etc in txt files & \\
    \hline 
\end{tabular}


\newpage

\begin{tabular}{| p{6.5cm} | p{6.5cm} | p{6.5cm} |}
    \hline
    code & meaning & information\\
    \hline 
    pandoc DOCUMENT.md -o DOCUMENT.pdf & compiles Markdowndocument to pdf & \\
    \hline
    idevicepair pair & pairs iphone &\\
    \hline
    ifuse /medie/iphone & mounts iphone to that path &\\
    \hline
    ifuse -u /medie/iphone & unmount iphone & \\
    \hline 
    convert -resize 50\% picture.jpg newpicture.jpg & picture will downsized to 50\% &\\
    \hline
    tar -xf file.tar.gz & unzips tar or gz files &\\
    \hline
    xclip -o > /path/to/file.txt & to put something from clipboard to a file &\\
    \hline 
    sudo snap remove package\_name & removes snap package &\\
    \hline
    sudo systemctl status spotifyd.service  & looks if spotifyd.service programm is active &\\
    \hline 
    spotifyd & start spotifyd api & after that spt will work \\
    \hline
    Ctrl + Z & suspends process & \\
    \hline 
    jobs & shows what processes are in the background & \\
    \hline
    fg \%number & brings process in foreground & number is the number shown when jobs executed \\
    \hline
    jupyter nbconvert --to pdf saturnclou1.ipynb & converts jupyter files to pdf & \\
    \hline 
    shred -u & removes not only pointer (rm) to data, it also removes the actual data & \\
    \hline
    fzf & a programm to find paths of files & \\
    \hline 
    \%lsblk -p | grep "disk | part" & to find devices that are mountd an can be & \\
    \hline 
    pdfseparate -f 1 -l 5 input.pdf output-page\%d.pdf & seperates pdf, 1 and 5 stands for the start and end page & \\
    \hline
    pdftk inputfile.pdf cat 1-5 output nameoutputfile.pdf & 1-5 range of pages you want from original pdf, cut pdf & \\
    \hline
    top & shows ongion processes with Memory ussage and others & \\
    \hline
    bashtop & show ongion process but better then top & \\
    \hline
    free -h & show ussage of RAM with buffer, swap und caches ussage & \\
    \hline 
    dd & copy cut file & in ranger \\
    \hline
    pp & paste what was cutted & in ranger \\
    \hline
    sudo nmcli dev wifi list & shows list of accessable wifis &\\
    \hline
    sudo nmcli --ask dev wifi connect network-ssid & connect to wifi with name network-ssid and ask for passwort & https://www.makeuseof.com/connect-to-wifi-with-nmcli/ \\
    \hline
    sensors & shows heat of elements & \\
    \hline
    ffmpeg –i example.mp4 –c:a copy –c:v copy –map\_metadata -1 example\_modified.mp4 >\& /dev/null & removes metadata from video & have not used like this \\
    \hline
    exiftool -all= file & removes all metadat from file & \\
    \hline 
    exiftool file & shows metadat & \\
    \hline
    psql -h [HOSTNAME] -p [PORT] -U [USERNAME] -W -d [DATABASENAME] & to conecct to database by psql & \\
    \hline
    newgrp docker & to log in into groupe (permission) docker where i can execute docker without sudo &\\
    \hline
    act & runs github actions localy with docker & need .github/workflows/file.yml \\
    \hline
    find /path/to/directory -type f -size +1G & finds files bigger the 1GB in path &\\
    \hline
    ollama run llama3.2 & runs ollama AI &\\
    \hline
    docker run -it ubuntu bash & downloads newest ubuntu image, runs it in bash & \\
    \hline 
    veracrypt --mount <<fileToMount>>  & mounts file to path & 
    recommended /media/liquid/vera \\
    \hline 
    veracrypt -d & umounts file & \\
    \hline
    uname -a & shows all the kernel infos & \\
    \hline 
    ip addr show & shows ip adress of all devices & can add specific name of device \\
    \hline 
    du -sh directory & shows size of directory & \\
    \hline
\end{tabular}


\begin{tabular}{| p{6.5cm} | p{6.5cm} | p{6.5cm} |}
    \hline
    \multicolumn{3}{|c|}{TMUX} \\
    \hline
    prefix = CRTL + a & this is the current prefix &\\
    \hline
    prefix = CRTL + a & this is the current prefix &\\
    \hline
    crtl + space & regular prefix for everything in tmux & \\
    \hline
    tmux & written in terminal, opens a new tmux session with name 0 & \\
    \hline
    tmux new-session -s name & opens new session with "name" & \\
    \hline 
    tmux ls & list active tmux sessions on server & \\
    \hline
    tmux attach-session -t name & attaches session with set name & \\
    \hline 
    prefix + c & opens new window in session & \\
    \hline
    prefix + d & detaches session & brings it in the backround \\
    \hline
    prefix + shift + \% & splits screen verticaly with new pane to the right & \\
    \hline
    prefix + shift + " & " " horrizontaly " " ont the buttom & \\
    \hline 
    prefix + x & closes pane\/session & \\
    \hline 
    prefix + q & shows nuber for pane & \\
    \hline 
    prefix + q + number of pane & changes to set pane & \\
    \hline
    prefix + number & changes to window with set number & \\
    \hline
    prefix + z & overlases your pane ocver the others & \\
    \hline
    prefix + s & shows open terminals and can change there & \\
    \hline
    prefix + : & enter commandline for tmux & \\
    \hline
    : new & opens new Session & \\
    \hline
    : new -s "name" & " " with name as name of session & \\
    \hline
    prefix + \$ & rename the session & \\
    \hline
    prefix + ; & Toggle last active pane, goes to last active pane &\\
    \hline
    Alt + H or L & moves between windows & \\
    \hline 
    prefix + ä & loads configs new & \\
    \hline
    prefix + Ctrl + s & safes tmux sessions & \\
    \hline 
    prefix + Ctrl + r & restore safe sessions & \\
        \hline
\end{tabular}
 :means in Commandmode of tmux
\newpage

\begin{tabular}{| p{6.5cm} | p{6.5cm} | p{6.5cm} |}
    \hline
    \multicolumn{3}{|c|}{RANGER} \\
    \hline
    :open\_with programm & opens file I hover with programm I want it to open  the file & \\
    \hline
    m <key> & marks directory as bookmark, key can be anything & \\
    \hline
    ' <key> & gets u to bookmark & \\
    \hline
\end{tabular}

\newpage

\centering
\begin{tabular}{| p{6.5cm} | p{6.5cm} | p{6.5cm} |}
    \hline
    \multicolumn{3}{|c|}{Bash "Syntax"} \\
    \hline
    code & meaning & information\\
    \hline
    \$"something" & with \$ u access something & \$1 gets first input\\
    \hline
    ("Function") & inside () something like find or pwd gos &\\
    \hline
    {"Command"} & Command stands for like 1\#\#*/ & in compileInfo it cuts the filename \\
    \hline
    \${parameter\#pattern} & shortest match from pattern in parameter from beginning is deleted &\\
    \hline 
    \${parameter\#\#pattern} & longest match from pattern in parameter from beginning is deleted & \${1\#\#*/} deletes everything from input bevor the last / \\
    \hline
    \${parameter\%pattern} & like \# just from end &\\
    \hline 
    \${parameter\%\%pattern} & like \#\# just from the end &\\
    \hline 
    " text " & everything in "" will be like String and no commands except \$,',backslash & echo "pdflatex textdummit.tex" output pdflatex textdummi.tex \\
    \hline
    ' command/text ' & similar to "" but can use variables & echo "my variable value is '\$variable' " \\
    \hline 
    variable=5 & declaring variable as integer 5 & works like in python \\
    \hline 
    == != + - * / \% & operator like in java but has also other use &\\
    \hline 
\end{tabular}

  \lstset{language=sh}
  \begin{lstlisting} 
  if [condition]
  then
      statement 
  fi

  if [condition 1]
  then 
      Statement 1
  elif [condition 2]
  then
      Statement 2
  else 
      Statement 3
  fi

  \end{lstlisting}

* \~/schulmaterial/efi/EigeneProjekte/inputoutputflanagan \\
** \~/schulmaterial/efi/Eigene\_Projekte/inputOutputFlanagan/path-setup 

for more information of syntax
{\url{https://www.gnu.org/software/bash/manual/bash.html#index-insert_002dcomment-_0028M_002d_0023_0029}}

{\url{https://www.tutorialspoint.com/unix/unix-basic-operators.htm}}

Samsung Electronics Co., Ltd GT-I9100 Phone [Galaxy S II] (Download mode)

Problems with PUBLIC-keys on ppa, -> just download ppa again like "sudo add-apt-repository ppa:neovim-ppa/stable" 

after "apt update" u see the repositorys and if they have launchpad in it u can short the ppas easy to there names. 

like "http://ppa.launchpad.net/neovim-ppa/stable/ubuntu" - "ppa:neovim-ppa/stable"



\end{document}
